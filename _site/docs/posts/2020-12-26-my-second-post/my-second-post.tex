\documentclass[]{tufte-handout}

% ams
\usepackage{amssymb,amsmath}

\usepackage{ifxetex,ifluatex}
\usepackage{fixltx2e} % provides \textsubscript
\ifnum 0\ifxetex 1\fi\ifluatex 1\fi=0 % if pdftex
  \usepackage[T1]{fontenc}
  \usepackage[utf8]{inputenc}
\else % if luatex or xelatex
  \makeatletter
  \@ifpackageloaded{fontspec}{}{\usepackage{fontspec}}
  \makeatother
  \defaultfontfeatures{Ligatures=TeX,Scale=MatchLowercase}
  \makeatletter
  \@ifpackageloaded{soul}{
     \renewcommand\allcapsspacing[1]{{\addfontfeature{LetterSpace=15}#1}}
     \renewcommand\smallcapsspacing[1]{{\addfontfeature{LetterSpace=10}#1}}
   }{}
  \makeatother

\fi

% graphix
\usepackage{graphicx}
\setkeys{Gin}{width=\linewidth,totalheight=\textheight,keepaspectratio}

% booktabs
\usepackage{booktabs}

% url
\usepackage{url}

% hyperref
\usepackage{hyperref}

% units.
\usepackage{units}


\setcounter{secnumdepth}{-1}

% citations


% pandoc syntax highlighting

% longtable

% multiplecol
\usepackage{multicol}

% strikeout
\usepackage[normalem]{ulem}

% morefloats
\usepackage{morefloats}


% tightlist macro required by pandoc >= 1.14
\providecommand{\tightlist}{%
  \setlength{\itemsep}{0pt}\setlength{\parskip}{0pt}}

% title / author / date
\title{The history of mathematical finance}
\author{true}
\date{12-26-2020}


\begin{document}

\maketitle




\hypertarget{louise-bachelier}{%
\section{Louise Bachelier}\label{louise-bachelier}}

한글은 어떤가 The history of stochastic integration and the modelling of
risky asset prices both begin with Brownian motion, so let us begin
there too. The earliest attempts to model Brownian motion mathematically
can be traced to three sources, each of which knew nothing about the
others: the first was that of T. N. Thiele of Copenhagen, who
effectively created a model of Brownian motion while studying time
series in 1880 {[}81{]}.2; the second was that of L. Bachelier of Paris,
who created a model of Brownian motion while deriving the dynamic
behavior of the Paris stock market, in 1900 (see, {[}1, 2, 11{]}); and
the third was that of A. Einstein, who proposed a model of the motion of
small particles suspended in a liquid, in an attempt to convince other
physicists of the molecular nature of matter, in 1905
\href{See\%20\%5B64\%5D\%20for\%20a\%20discussion\%20of\%20Einstein’s\%20model\%20and\%20his\%20motivations.}{21}
Of these three models, those of Thiele and Bachelier had little impact
for a long time, while that of Einstein was immediately influential. We
go into a little detail about what happened to Bachelier, since he is
now seen by many as the founder of modern Mathematical Finance. Ignorant
of the work of Thiele (which was little appreciated in its day) and
preceding the work of Einstein, Bachelier attempted to model the market
noise of the Paris Bourse. Exploiting the ideas of the Central Limit
Theorem, and realizing that market noise should be without memory, he
reasoned that increments of stock prices should be independent and
normally distributed. He combined his reasoning with the Markov property
and semigroups, and connected Brownian motion with the heat equation,
using that the Gaussian kernel is the fundamental solution to the heat
equation.

\hypertarget{albert-einstein}{%
\section{Albert Einstein}\label{albert-einstein}}

Let us now turn to Einstein's model. In modern terms, Einstein assumed
that Brownian motion was a stochastic process with continuous paths,
independent increments, and stationary Gaussian increments. He did not
assume other reasonable properties (from the standpoint of physics),
such as rectifiable paths. If he had assumed this last property, we now
know his model would not have existed as a process. However, Einstein
was unable to show that the process he proposed actually did exist as a
mathematical object. This is understandable, since it was 1905, and the
ideas of Borel and Lebesgue constructing measure theory were developed
only during the first decade of the twentieth century. In 1913 Daniell's
approach to measure theory (in which integrals are defined before
measures) appeared, and it was these ideas, combined with Fourier
series, that N. Wiener used in 1923 to construct Brownian motion,
justifying after the fact Einstein's approach. Indeed, Wiener used the
ideas of measure theory to construct a measure on the path space of
continuous functions, giving the canonical path projection process the
distribution of what we now know as Brownian motion. Wiener and others
proved many properties of the paths of Brownian motion, an activity that
continues to this day. Two key properties relating to stochastic
integration are that (1) the paths of Brownian motion have a non zero
finite quadratic variation, such that on an interval (s, t), the
quadratic variation is (t−s) and (2) the paths of Brownian motion have
infinite variation on compact time intervals, almost surely. The second
property follows easily from the first. Note that if Einstein were to
have assumed rectifiable paths, Wiener's construction would have
essentially proved the impossibility of such a model. In recognition of
his work, his construction of Brownian motion is often referred to as
the Wiener process. Wiener also constructed a multiple integral, but it
was not what is known today as the ``Multiple Wiener Integral'': indeed,
it was K. Itˆo, in 1951, when trying to understand Wiener's papers (not
an easy task), who refined and greatly improved Wiener's ideas {[}36{]}.

\hypertarget{uxc6b0uxc7a5uxcd98}{%
\section{우장춘}\label{uxc6b0uxc7a5uxcd98}}

Let us now turn to Einstein's model. In modern terms, Einstein assumed
that Brownian motion was a stochastic process with continuous paths,
independent increments, and stationary Gaussian increments. He did not
assume other reasonable properties (from the standpoint of physics),
such as rectifiable paths. If he had assumed this last property, we now
know his model would not have existed as a process. However, Einstein
was unable to show that the process he proposed actually did exist as a
mathematical object. This is understandable, since it was 1905, and the
ideas of Borel and Lebesgue constructing measure theory were developed
only during the first decade of the twentieth century. In 1913 Daniell's
approach to measure theory (in which integrals are defined before
measures) appeared, and it was these ideas, combined with Fourier
series, that N. Wiener used in 1923 to construct Brownian motion,
justifying after the fact Einstein's approach. Indeed, Wiener used the
ideas of measure theory to construct a measure on the path space of
continuous functions, giving the canonical path projection process the
distribution of what we now know as Brownian motion. Wiener and others
proved many properties of the paths of Brownian motion, an activity that
continues to this day. Two key properties relating to stochastic
integration are that (1) the paths of Brownian motion have a non zero
finite quadratic variation, such that on an interval (s, t), the
quadratic variation is (t−s) and (2) the paths of Brownian motion have
infinite variation on compact time intervals, almost surely. The second
property follows easily from the first. Note that if Einstein were to
have assumed rectifiable paths, Wiener's construction would have
essentially proved the impossibility of such a model. In recognition of
his work, his construction of Brownian motion is often referred to as
the Wiener process. Wiener also constructed a multiple integral, but it
was not what is known today as the ``Multiple Wiener Integral'': indeed,
it was K. Itˆo, in 1951, when trying to understand Wiener's papers (not
an easy task), who refined and greatly improved Wiener's ideas {[}36{]}.



\end{document}
